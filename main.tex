%%%%%%%%%%%%%%%%%%%%%%%%%%%%%%%%%%%%%%%%%
% Beamer Presentation - LaTeX Template
% Version 2.0 (March 8, 2022)
% Original Template: https://www.LaTeXTemplates.com
% License: CC BY-NC-SA 4.0
%%%%%%%%%%%%%%%%%%%%%%%%%%%%%%%%%%%%%%%%%

\documentclass[
  11pt, % Default font size
  aspectratio=169, % Set 16:9 aspect ratio
]{beamer}

% Packages
\usepackage[T1]{fontenc}
\usepackage{subcaption}
\usepackage{tikz}
\usepackage{url}
\usepackage[base]{babel} % comment if you intend to create a presentation in french
% \usepackage[french]{babel} % uncomment if you intend to create a presentation in french

% Layout theme
\usetheme[
  % showmaxslides, % displays the total page number in the page numbering
  showsectiontoc, % displays a table of contents at the beginning of each section
]{Aramis}

\AtBeginSubsection[]{
  \begin{frame}{Section outline}
    \tableofcontents[
      sectionstyle= show/shaded,
      subsectionstyle=show/shaded/hide
    ]
  \end{frame}
}

% Paths and misc stuff
\graphicspath{{img/}}

%------------------
%  PRESENTATION INFORMATION
%------------------

\title{\texttt{tmux} :\newline an open-source and keyboard-centric terminal multiplexer}
\subtitle{With some typing tips to skyrocket your \newline development workflow}
\author{Léo Guillon}
\date{09/12/2025}

%------------------

% \includeonlyframes{titlepage}

\begin{document}

%------------------
%  TITLE SLIDE
%------------------

\begin{frame}[plain, label=titlepage]
  \titlepage % Displays the title slide
\end{frame}

%------------------
%  TABLE OF CONTENTS SLIDE
%------------------

\begin{frame}
  \frametitle{Presentation Outline} % Slide title
  \tableofcontents[hideallsubsections, pausesections] % Displays the table of contents
\end{frame}

%------------------
%  BODY SECTIONS
%------------------

\section{Introduction}

\begin{frame}{Who am I ? What I work on ?}

  % TODO: photo CV

  \begin{columns}
    \begin{column}{0.7\textwidth}

      \begin{description}
        \item[dec. 2024 → nov. 2025] Research engineer @ Aramis
        \item[sep. 2025 → june 2026] Preparing mathematics aggregation @ Sorbonne Université
      \end{description}
    \end{column}
    \begin{column}{0.3\textwidth}
      \begin{figure}
        \centering
        \includegraphics[width=.7\textwidth]{img/cv_icm.jpg}
      \end{figure}

    \end{column}
  \end{columns}

  \vfill
  \pause

  Dev needs :
  \begin{itemize}
    \item Code : R, Julia, Python
    \item Note-taking
    \item Document writing : \LaTeX
  \end{itemize}

\end{frame}

\begin{frame}{My general work and tools philosophy}
  My work philosophy is to use tools that are :
  \begin{itemize}
    \item<2-> \strong{efficient :} keyboard-centric, local files, …
    \item<3-> \strong{minimalist :} ability to just do what I need, nothing more ; customization if possible
    \item<4-> \strong{open :} open-source if possible
  \end{itemize}

  \vfill

  \begin{columns}<5->
    \begin{column}{0.5\textwidth}

      Examples:
      \begin{itemize}
        \item Zen Browser (web browser);
        \item Obsidian (note-taking);
        \item Sioyek (pdf reader);
        \item WezTerm (terminal), alongside:
          \begin{itemize}
            \item \texttt{neovim} (code);
            \item \texttt{yazi} (file explorer)
            \item \texttt{tmux} (terminal multiplexer);
          \end{itemize}
      \end{itemize}
    \end{column}
    \begin{column}{0.4\textwidth}
      \begin{figure}
        \centering
        \includegraphics[height=.45\textheight]{img/logos.png}
      \end{figure}
    \end{column}
  \end{columns}

  \vfill

\end{frame}

\section{\texttt{tmux} as your terminal multiplexer}

\subsection{Use cases}

\begin{frame}{What is a terminal multiplexer ?}

  % TODO: block

  General idea : instead of working with single terminal, and nuking everything once you close it, you split it in multiple windows, as you can do in VSCode for example.

  % TODO: insérer une image d’exemple, comparaison entre les deux

\end{frame}

\begin{frame}{Use cases for a terminal multiplexer}

  If you :
  \begin{itemize}
    \item<2-> are already used to the terminal and its commands;
    \item<3-> use several command line interface (CLI) and terminal user interface (TUI) tools;
    \item<4-> want to make your terminal sessions persistent;
    \item<5-> are willing to have a streamlined, keyboard-focused workflow,
  \end{itemize}

  \vfill

  \begin{columns}<6->
    \begin{column}{0.65\textwidth}

      then, a terminal multiplexer may be the right tool for your day-to-day dev work environment !
    \end{column}
    \begin{column}{0.3\textwidth}
      \begin{figure}
        \centering
        \includegraphics[height=.35\textheight]{img/thumb up.jpg}
      \end{figure}

    \end{column}
  \end{columns}

\end{frame}

\subsection{General workflow and vocabulary}

\begin{frame}{\texttt{tmux}’s hierarchy}

  \begin{description}
    \item[pane]<2-> a single terminal instance
    \item[window]<3-> a collection of panes, arranged in a defined layout
    \item[session]<5-> a collection of windows
  \end{description}
  \begin{columns}
    \begin{column}{0.5\textwidth}<4->

      \begin{figure}
        \centering
        \includegraphics[height=0.45\textheight]{img/panes.png}
      \end{figure}
    \end{column}
    \begin{column}{0.5\textwidth}<6->

      \begin{figure}{}
        \centering
        \includegraphics[width=\textwidth]{img/statusbar.png}
      \end{figure}
    \end{column}
  \end{columns}

\end{frame}

\begin{frame}{Workflow with tmux}

  \begin{enumerate}
    \item Terminal startup
    \item You attach to an already existing tmux session, or create a new one
    \item In a session, you go in your working window (or create a new one)
    \item In a window, you choose or create a pane
    \item Have a nice working session !
  \end{enumerate}

  \vfill
  \pause

  Once you’re done :
  \begin{enumerate}
    \item You can kill your pane, window and/or session if you want to
    \item Detach from your session, and done !
  \end{enumerate}

\end{frame}

\begin{frame}{Example with my personal use}

  Use case for my current work :

  % TODO: custom the blocks

  \begin{block}{\LaTeX custom commands}
    Writing a \LaTeX~package for my custom mathematical commands, as well as a \LaTeX~document with a big table to present them, as well as a \texttt{.json} file to have autocompletion snippets in Obsidian.
  \end{block}

\end{frame}

\subsection{Setup and commands}

\begin{frame}{\texttt{tmux}’s config}

  \texttt{tmux} works with a config file, \texttt{tmux.conf}, in your \texttt{~/.config/tmux} folder

  You can also split up the config as you want, to make it more modular

  % TODO: écrire avec l’outil sur navigateur pour écrire un arbre

  % TODO: capture de l’arborescence de mon setup

\end{frame}

\begin{frame}{\texttt{tmux}’s commands startup point : the prefix key}
  Different philosophy for commands : instead of having keyboard shortcuts as keys combination, inputs are keys \emph{succession}.

  \begin{itemize}
    \item not compete with usual OS or app keyboard shortcuts, TUI specific commands
    \item it’s easier to enter a succession than a combination of keystrokes
  \end{itemize}

  \pause

  \[\text{\texttt{tmux} command} = \text{prefix}+\text{command keystroke}\]

\end{frame}

\begin{frame}{The default prefix, and the better prefix}

  Default prefix : \texttt{C-b}

  \begin{figure}
    \centering
    \includegraphics<1-2>[width=.65\textwidth]{img/prefix-1.png}
    \includegraphics<3->[width=.65\textwidth]{img/prefix-2.png}
  \end{figure}

  \onslide<2->

  My recommandation : setup \texttt{C-} on Caps Lock (in your MacBook settings), and change it to \texttt{C-a} (on a QWERTY layout) or \texttt{C-q} (on a AZERTY layout)
\end{frame}

\begin{frame}[fragile]{The better prefix: how to}

  Caps lock to Control :

  Settings → keyboard → keyboard shortcuts → modifier keys

  % TODO: insérer image d’exemple
  \vfill
  \pause

  In \texttt{tmux.conf} :

  \begin{verbatim}
    set-option -g prefix C-a
    unbind C-b
    bind C-a send-prefix
  \end{verbatim}

\end{frame}

\begin{frame}{Commands worth considering to begin with}

  % TODO: diviser en plusieurs diapos

  \begin{description}
    \item[d] Detach current session
    \item["] Horizontal split
    \item[] Vertical split
    \item[c] Create a new window
    \item[hjkl/←↓↑→] move from pane to pane (Vim logic for arrows)
      % TODO: ajouter une image pour justifier le positionnement hjkl
    \item[1–9] move to window of the corresponding number
    \item[n/b] move to next/previous window
    \item[s] list Sessions
    \item[,] rename current window
    \item[\$] rename current session
    \item[:] open \texttt{tmux}’s command line
    \item[?] list all key binding
  \end{description}

\end{frame}

\begin{frame}{Useful settings}

  How to change or add a keybinding : …

  % TODO: add how to add a keybinding

  \begin{itemize}
    \item \strong{Enabling the mouse :} \texttt{:set -g mouse on}
    \item \strong{Start windows indexs at 1 :} \texttt{set -g base-index 1}
    \item \strong{Automatically renumber windows :} \texttt{set -g renumber-windows on}
    \item \strong{Stop windows auto-renaming :} \texttt{set -g allow-rename off}
  \end{itemize}

\end{frame}

\begin{frame}{Bonus setup : \emph{plugins}}

  \begin{itemize}
    \item \textbf{\texttt{tpm}} \texttt{tmux} plugin manager, mandatory
    \item \textbf{\texttt{tmux-continuum}+\texttt{tmux-resurrect}} allows saving and resurrection of sessions even when your machine shuts down
    \item \textbf{\texttt{vim-tmux navigator}} seamless navigation between \texttt{tmux} panes and \texttt{vim} panes
    \item lot of others plugins, and color themes !
  \end{itemize}

  % TODO: add several vskip

  % TODO: encadré

  General advice : start small, and add relevant keybindings and plugins for your specific workflow.

\end{frame}

\section{Keyboard tips}

\subsection{Touch-typing}

\begin{frame}{\emph{Hunt and peck} vs. \emph{touch-typing}}
  \pause
  \begin{description}
    \item[hunt and peck] looking at your keys, and searching them with your eyes
    \item[touch-typing] one key = one finger, typing without looking at your keyboard
  \end{description}

  \begin{figure}
    \centering
    \includegraphics[width=.6\textwidth]{img/huntandpeck.png}
  \end{figure}

\end{frame}

\begin{frame}{Touch-typing : general posture}

  % TODO: posture ergonomique générale

  \begin{figure}
    \centering
    \includegraphics[width=.8\textwidth]{img/posture.png}
  \end{figure}

  % TODO: image des dix doigts en position de repos

\end{frame}

\begin{frame}{Touch-typing : advices}

  % TODO: ajouter une pause entre chaque item
  \begin{itemize}
    \item<2-> Type with your 10 fingers.
    \item<3-> One key = one finger.
    \item<4-> Never look at the keyboard, mask it if needed
    \item<5-> Always come back to the default position
    \item<6-> \strong{Accuracy over speed !!} Aim for ≥96\% accuracy, speed will build up over time.
    \item<7-> Error : use opt+backspace (or ctrl+backspace on Windows) to delete an entire word.
  \end{itemize}
\end{frame}

\begin{frame}{How to practice touch-typing ?}
  On a day to day basis : apply previous slides principles.

  \vfill
  \onslide<2->
  Websites to practice :
  \begin{enumerate}
    \item<2-> \strong{\texttt{keybr} :}  to learn to touch-type with your keyboard layout
    \item<3-> \strong{\texttt{monkeytype} :} to improve your accuracy (first) and then your speed
  \end{enumerate}

  % TODO: capture d’écran keybr et monkeytype

  \begin{columns}
    \begin{column}{0.5\textwidth}<2->
      \begin{figure}
        \centering
        \includegraphics[width=.6\textwidth]{img/keybr.png}
      \end{figure}

    \end{column}
    \begin{column}{0.5\textwidth}<3->
      \begin{figure}
        \centering
        \includegraphics[width=.8\textwidth]{img/monkeytype.png}
      \end{figure}
    \end{column}
  \end{columns}

  \vfill

  \onslide<4->
  Other alternatives : 10fastfingers, typingrace, …
\end{frame}

\subsection{Learn your keyboard shortcuts !}

\begin{frame}{MacOS usual keyboard shortcuts}

\end{frame}

\subsection{Text expander}

\begin{frame}{What is a text expander ?}

  % TODO: résumé

\end{frame}

\begin{frame}{Espanso : an open-source text expander}

  % TODO: liste des particularités notables d’Espanso

\end{frame}

\begin{frame}{Example of practical uses}

  \begin{itemize}
    \item \strong{ligatures} : -> = →
    \item \strong{symbols} : :cmd:, :heart:
    \item \strong{word shortcuts} : cdlt = cordialement
    \item \strong{custom infos} : :mpro = \url{leo.guillon@mailo.com}
    \item \strong{automatic} : :today = current date
      % TODO: décider si ça vaut le coup de mettre les formulaires
      % \item \strong{forms} : :lbc
    \item \strong{useless, but fun} : :flip, …
  \end{itemize}

\end{frame}

\subsection{Some more rabbit holes}

\begin{frame}{Making your keyboard more ergonomic}

  Useful keys closer to the default position : enter, backspace, etc.

  Ctrl on Caps Lock is only the first step : imagine that you had both Ctrl \emph{and} esc on Caps Lock.

  Putting useful shortcuts

  Tons of concepts : layer-taps, double-taps, homerow mods, tap-hold, etc.

  Software and github repos : Karabiner (MacOS), Kanata (multi-platform), Arsenik/Miryoku/… for inspirations

  % TODO: image d’Arsenik ou Miryoku for inspiration

  \begin{figure}
    \centering
    \includegraphics[width=.6\textwidth]{img/arsenik.png}
    \caption{Arsenik keyboard, all options enabled}
  \end{figure}

\end{frame}

\begin{frame}{Changing your keyboard layout}

  % TODO: insérer ce texte dans un encadré

  General idea : putting the most frequent letters in your language under the most accessible keys according to the « default position »

  \vfill
  \pause

  \begin{description}
    \item[English] Qwerty → Dvorak, Colemak, …
    \item[French] Azerty → Bépo → Ergo-L (<3 <3 <3)
  \end{description}

  \vfill
  \pause

  \begin{columns}
    \begin{column}{0.5\textwidth}

      \begin{figure}
        \centering
        \includegraphics[width=\textwidth]{img/azerty.png}
        \caption{Azerty layout heatmap}
      \end{figure}
    \end{column}

    \pause
    \begin{column}{0.5\textwidth}
      \begin{figure}
        \centering
        \includegraphics[width=\textwidth]{img/ergol.png}
        \caption{Ergo-L layout heatmap}
      \end{figure}
    \end{column}
  \end{columns}

\end{frame}

\begin{frame}{Switching to an ergonomic keyboard}

  \center{Warning : \strong{Ultra-deep endless rabbit hole!}}

  % TODO: lapin blanc, montage avec des images des layouts
\end{frame}

\begin{frame}{Conclusion}

  % TODO: conclusion écrite

  % tmux : a nice tool to improve your terminal and dev workflow
  % keyboard efficiency : a must when you spend 8 hours a day behind a computer
  % practice touch-typing, and improve going forward
\end{frame}

\end{document}
