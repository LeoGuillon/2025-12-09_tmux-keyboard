%%%%%%%%%%%%%%%%%%%%%%%%%%%%%%%%%%%%%%%%%
% Beamer Presentation - LaTeX Template
% Version 2.0 (March 8, 2022)
% Original Template: https://www.LaTeXTemplates.com
% License: CC BY-NC-SA 4.0
%%%%%%%%%%%%%%%%%%%%%%%%%%%%%%%%%%%%%%%%%

\documentclass[
  11pt, % Default font size
  aspectratio=169, % Set 16:9 aspect ratio
]{beamer}

% Packages
\usepackage[T1]{fontenc}
\usepackage{subcaption}
\usepackage{tikz}
\usepackage{lipsum} % useful for creating some dummy text
\usepackage[base]{babel} % comment if you intend to create a presentation in french
% \usepackage[french]{babel} % uncomment if you intend to create a presentation in french

% Layout theme
\usetheme[
  % showmaxslides, % displays the total page number in the page numbering
  showsectiontoc, % displays a table of contents at the beginning of each section
]{Aramis}

% Paths and misc stuff
\graphicspath{{img/}}

%------------------
%  PRESENTATION INFORMATION
%------------------

\title{\texttt{tmux} : an open-source and keyboard-centric terminal multiplexer}
\subtitle{With some keyboard efficiency tips}
\author{Léo Guillon}
\date{09/12/2025}

%------------------

% \includeonlyframes{titlepage}

\begin{document}

%------------------
%  TITLE SLIDE
%------------------

\begin{frame}[plain, label=titlepage]
  \titlepage % Displays the title slide
\end{frame}

%------------------
%  TABLE OF CONTENTS SLIDE
%------------------

\begin{frame}
  \frametitle{Presentation Outline} % Slide title
  \tableofcontents[hideallsubsections] % Displays the table of contents
\end{frame}

%------------------
%  BODY SECTIONS
%------------------

\section{Introduction}

\begin{frame}{Who am I, and my general work philosophy}
  My work philosophy : use tools that are :
  \begin{description}
    \item[efficient]
    \item[minimalist]
    \item[open]
  \end{description}

  Examples of tools I use :
  \begin{itemize}
    \item Zen Browser
    \item Obsidian
    \item Sioyek
    \item Wezterm (terminal), with \texttt{neovim} (code), \texttt{yazi} (file explorer), \texttt{tmux}
  \end{itemize}

\end{frame}

\section{\texttt{tmux} as your terminal multiplexer}

\subsection{Use cases}

\begin{frame}{What is a terminal multiplexer ?}

\end{frame}

\subsection{General workflow and vocabulary}

\begin{frame}{\texttt{tmux}’s hierarchy}

\end{frame}

\begin{frame}{Workflow with tmux}

\end{frame}

\begin{frame}{Example with my personal use}

\end{frame}

\subsection{Commands and setup}

\section{Keyboard tips}

\subsection{Touch-typing}

\begin{frame}{\emph{Hunt and peck} vs. \emph{touch-typing}}

\end{frame}

\begin{frame}{Touch-typing : general posture}

\end{frame}

\begin{frame}{Touch-typing : advices}

\end{frame}

\subsection{Learn your keyboard shortcuts !}

\begin{frame}{MacOS usual keyboard shortcuts}

\end{frame}

\begin{frame}{App-specific shortcuts}

\end{frame}

\subsection{Text expander}

\begin{frame}{What is a text expander ?}

\end{frame}

\begin{frame}{Espanso : an open-source text expander}

\end{frame}

\subsection{Some more rabbit holes}

\begin{frame}{Some more rabbit holes}

\end{frame}

\section*{Conclusion}

\begin{frame}{Conclusion}
  % tmux : a nice tool to improve your terminal and dev workflow
  % keyboard efficiency : a must when you spend 8 hours a day behind a computer
  % practice touch-typing, and improve going forward
\end{frame}
\end{document}
