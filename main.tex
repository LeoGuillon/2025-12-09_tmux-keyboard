%%%%%%%%%%%%%%%%%%%%%%%%%%%%%%%%%%%%%%%%%
% Beamer Presentation - LaTeX Template
% Version 2.0 (March 8, 2022)
% Original Template: https://www.LaTeXTemplates.com
% License: CC BY-NC-SA 4.0
%%%%%%%%%%%%%%%%%%%%%%%%%%%%%%%%%%%%%%%%%

\documentclass[
  11pt, % Default font size
  aspectratio=169, % Set 16:9 aspect ratio
]{beamer}

% Packages
\usepackage[T1]{fontenc}
\usepackage{subcaption}
\usepackage{tikz}
\usepackage{lipsum} % useful for creating some dummy text
\usepackage[base]{babel} % comment if you intend to create a presentation in french
% \usepackage[french]{babel} % uncomment if you intend to create a presentation in french

% Layout theme
\usetheme[
  % showmaxslides, % displays the total page number in the page numbering
  showsectiontoc, % displays a table of contents at the beginning of each section
]{Aramis}

% Paths and misc stuff
\graphicspath{{img/}}

%------------------
%  PRESENTATION INFORMATION
%------------------

\title{\texttt{tmux}\newline an open-source terminal multiplexer}
\subtitle{With some keyboard tips to skyrocket your development workflow}
\author{Léo Guillon}
\date{09/12/2025}

%------------------

% \includeonlyframes{titlepage}

\begin{document}

%------------------
%  TITLE SLIDE
%------------------

\begin{frame}[plain, label=titlepage]
  \titlepage % Displays the title slide
\end{frame}

%------------------
%  TABLE OF CONTENTS SLIDE
%------------------

\begin{frame}
  \frametitle{Presentation Outline} % Slide title
  \tableofcontents[hideallsubsections] % Displays the table of contents
\end{frame}

%------------------
%  BODY SECTIONS
%------------------

\section{Introduction}

\begin{frame}{Who am I, and my general work philosophy}
  My work philosophy : use tools that are :
  \begin{itemize}
    \item \textbf{efficient}
    \item \textbf{minimalist}
    \item \textbf{open}
  \end{itemize}

  Examples of tools I use :
  \begin{itemize}
    \item Zen Browser
    \item Obsidian
    \item Sioyek
    \item Wezterm (terminal), with \texttt{neovim} (code), \texttt{yazi} (file explorer), \texttt{tmux}
  \end{itemize}

\end{frame}

\section{\texttt{tmux} as your terminal multiplexer}

\subsection{Use cases}

\begin{frame}{What is a terminal multiplexer ?}

  General idea : instead of working with single terminal, and nuking everything once you close it, you split it in multiple windows, as you can do in VSCode for example.

\end{frame}

\begin{frame}{Use cases for a terminal multiplexer}

  If you :
  \begin{itemize}
    \item are already used to the terminal and its commands ;
    \item use several command line interface (CLI) and terminal user interface (TUI) tools, and simultaneously ;
    \item make your terminal sessions persistant even when shutting down ;
    \item are willing to have a streamlined, keyboard-focused workflow,
  \end{itemize}

  then, a terminal multiplexer may be the right tool for your day-to-day dev work environment !

\end{frame}

\subsection{General workflow and vocabulary}

\begin{frame}{\texttt{tmux}’s hierarchy}

  % idée : mettre une pyramide

  \begin{description}
    \item[pane] a single terminal instance
    \item[window] a collection of panes, arranged in a defined layout (similar to a tab)
    \item[session] a collection of windows
  \end{description}

\end{frame}

\begin{frame}{Workflow with tmux}

  \begin{enumerate}
    \item Terminal startup
    \item You attach to an already existing tmux session, or create a new one
    \item In a session, you go in your working window (or create a new one)
    \item In a window, you choose or create a pane
    \item Have nice working session !
  \end{enumerate}

  Once you’re done :
  \begin{enumerate}
    \item You can kill your pane, window and/or session if you want to
    \item Detach from your session, and done !
  \end{enumerate}

\end{frame}

\begin{frame}{Example with my personal use}

\end{frame}

\subsection{Commands and setup}

\begin{frame}{\texttt{tmux}’s config}

  \texttt{tmux} works with a config file, \texttt{tmux.conf}, in your \texttt{~/.config/tmux} folder

  You can also split up the config as you want, to make it more modular

  % TODO: capture de l’arborescence de mon setup

\end{frame}

\begin{frame}{\texttt{tmux}’s commands startup point : the prefix key}
  Different philosophy for commands : instead of having keyboard shortcuts as keys combination, inputs are keys \emph{succession}.

  \begin{itemize}
    \item not compete with usual OS or app keyboard shortcuts, TUI specific commands
    \item it’s easier to enter
  \end{itemize}

  % \[\text{\texttt{tmux} command} = \text{prefix}+\text{command keystroke}\]

  % Default prefix : \texttt{^B}
  %
  % My recommandation : setup \texttt{^} on ⇪, and change it to \texttt{^A} (on a QWERTY layout)

\end{frame}

\begin{frame}{Commands worth considering to begin with}

  \begin{description}
    \item[d] Detach current session
    \item[c] Create a new window
    \item[C] Create a new session with a window
    \item[hjkl/←↓↑→] move from pane to pane (Vim logic for arrows)
      % TODO: ajouter une image pour justifier le positionnement hjkl
    \item[1–9] move to window of the corresponding number
    \item[s] list Sessions
    \item[,] rename current window
    \item[\$] rename current session
    \item[:] open \texttt{tmux}’s command line
  \end{description}

\end{frame}

\begin{frame}{Bonus setup : \emph{plugins}}

  \begin{itemize}
    \item \textbf{\texttt{tpm}} \texttt{tmux} plugin manager, mandatory
    \item \textbf{\texttt{tmux-continuum}+\texttt{tmux-resurrect}} allows saving and resurrection of sessions even when your machine shuts down
    \item \textbf{\texttt{vim-tmux navigator}} seamless navigation between \texttt{tmux} panes and \texttt{vim} panes
    \item lot of others plugins
  \end{itemize}

  My config is available at …

\end{frame}

\section{Keyboard tips}

\subsection{Touch-typing}

\begin{frame}{\emph{Hunt and peck} vs. \emph{touch-typing}}

\end{frame}

\begin{frame}{Touch-typing : general posture}

\end{frame}

\begin{frame}{Touch-typing : advices}

\end{frame}

\subsection{Learn your keyboard shortcuts !}

\begin{frame}{MacOS usual keyboard shortcuts}

\end{frame}

\begin{frame}{App-specific shortcuts}

\end{frame}

\subsection{Text expander}

\begin{frame}{What is a text expander ?}

\end{frame}

\begin{frame}{Espanso : an open-source text expander}

\end{frame}

\subsection{Some more rabbit holes}

\begin{frame}{Some more rabbit holes}

\end{frame}

\section*{Conclusion}

\begin{frame}{Conclusion}
  % tmux : a nice tool to improve your terminal and dev workflow
  % keyboard efficiency : a must when you spend 8 hours a day behind a computer
  % practice touch-typing, and improve going forward
\end{frame}
\end{document}
